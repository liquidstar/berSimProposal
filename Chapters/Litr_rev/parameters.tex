\section{Communication Link Parameters}
The dominant consideration in the design of digital communication systems is their ability to adequately out-perform certain marginal metrics over a non-ideal channel\cite{hayk}. These metrics of System Performance are:

%----------------------------------
%	PAPR Theoretical coverage
%----------------------------------
\subsection{Peak to Average Power Ratio (\gls{PAPR})}
Multi carrier modulations like \gls{OFDM} exhibit large dynamic range because of the strong power fluctuations that can be quantified as \gls{PAPR}. \gls{PAPR} is the relation between maximum power of a sample in a given \gls{OFDM} transmit \gls{symbol} divided by the average power of that \gls{OFDM} \gls{symbol}\cite{french}. The \gls{OFDM} signal is a sum of $N$ independent complex random variables each of which can be considered a signal of a different carrier frequency. In the most extreme case, the different carriers may all line up in phase at some instance in time, producing an amplitude peak equal to the sum of amplitudes of the individual carriers.

\gls{PAPR} of the transmitted \gls{OFDM} signal has large peaks that introduce a serious degradation in performance when the signal passes through a nonlinear \gls{HPA}. The non-linearity of \gls{HPA}s leads to in-band distortion which increases \gls{BER}, and out-of-band radiation, which causes adjacent channel interference\cite{prob}. The subcarriers are added constructively to form large peaks. High peak power requires an expensive \gls{HPA}, \gls{ADC} and \gls{DAC}. Peaks are distorted non-linearly due to amplifier imperfection in \gls{HPA}s. 

If the amplifier operates in its non-linear region, out of band and in-band spectrum radiations are produced which appears as adjacent channel interference\cite{gaurav}. Moreover, without large power back-offs, out of band power cannot be kept within specified limits, leading to inefficient amplification and expensive transmitter amplifiers.
 
The problem of high peak amplitude excursions is most severe at the transmitter output whereby to transmit the peaks sans clipping, not only must the \gls{DAC} have enough bits to accommodate the peaks, but the power amplifier must as well remain linear over an amplitude range that includes the peak amplitudes\cite{ofdm_intro}.

From \cite{papr_paper}, for the \gls{OFDM} system with $N$ carriers, the baseband with normalized power from an \gls{IFFT} operation is:

$$x(t) = \frac{1}{\sqrt{N}}\sum_{k=0}^{N-1}X_k\exp (jk\Delta ft)$$
$\Delta f$ is subcarrier spacing and $X_k$ is the spectrum of the $k$th sub-carrier.
From this, the \gls{PAPR} of an \gls{OFDM} signal can be derived as:
$$\text{PAPR} = \frac{\left(\left|x(t)\right|^2\right)_{\max}}{E\left\{\left|x(t)\right|^2\right\}}$$

\subsection{Average Signal-to-Noise Ratio (\gls{SNR})}
\gls{SNR} is defined as the ratio of the power of a signal to the power of background noise as measured at equivalent points in a system and within the same bandwidth.\cite{dcommoha}
$$SNR = \frac{P_{signal}}{P_{noise}}$$

For an effective transmission technology, retrieval of bit streams from a received waveform  should be as error free as possible, despite non-idealities of the communication system. This is done by achieving the best possible \gls{SNR} free from any \gls{ISI}. The \gls{SNR} can degrade through losses and/or increase in noise/interfering signal power. Losses occur when a portion of the signal is absorbed, diverted, scattered, or reflected along its route to the intended receiver.

According to \cite{AWGN}, the primary causes for error-performance degradation is:
\begin{itemize}
	\item Effect of filtering at the transmitter, channel, and receiver. For a nonideal system transfer with
\gls{ISI}.
	\item  Electrical noise and interference produced by a variety of sources, such as galaxy and
	atmospheric noise, switching transients, intermodulation noise, as well as interfering signals
	from other sources.
\end{itemize}
	
In the context of a communication system subject to fading, the more appropriate measure is average \gls{SNR}. This is the statistical averaging over the probability distribution of the fading. Mathematically, where $\gamma$ denotes instantaneous \gls{SNR}, then
$$\bar{\gamma} \triangleq \int_0^{\infty} \gamma P_{\gamma} \left(\gamma\right) d\gamma$$
is the average \gls{SNR}. $P_{\gamma} \left(\gamma\right)$ is the probability density function of $\gamma$.
\subsubsection{Outage probability}
When the \gls{SNR} level at the receiver falls below a certain threshold value the system is said to be in outage. Correct detection of signal may not be possible at these low \gls{SNR} levels. This results in a large number of bits and therefore \gls{symbol}s being received in error making the system unreliable\cite{MIMO}.

Outage probability is defined as the probability that the instantaneous output \gls{SNR}, $\gamma$, falls below a certain specified threshold $\gamma_{th}$\cite{dcommoha}. Mathematically expressed:
$$P_{out} = \int_0^{\gamma_{th}} P_{\gamma}\left(\gamma\right) d\gamma$$
This is the cumulative distribution function of $\gamma$.

%---------------------------------
%	System Complexity
%---------------------------------
\subsection{Complexity}
We can classify as system complexity and computational complexity:
\subsubsection*{System Complexity}
It refers to the number of circuits implemented and the technical difficulty associated with the system. This reflects upon the cost of manufacturing which is a key consideration when selecting a communication system. 

For communication systems,
there are various grounds of making a complexity comparison. The demodulator is usually more complex than the modulator with the coherent demodulator being more complex since it requires carrier recovery. Sophisticated algorithms like Viterbi are requires which makes a system even more complex.\cite{fuqin}

\subsubsection{Computational Complexity}
Communication complexity aims at studying the number of communication bits that the participants of a communication system need to exchange to perform certain tasks.  It studies problems which model typical communication needs of computations and attempts to determine the bounds on the amount of communication between processors that these problems require. For a collection of communication problems f , we can show interesting lower bounds $L_f$ on their communication complexity under different
types of protocols (one-way, multi-way, multi-party), that can be used to emulate different kind of
algorithmic primitives.


%-----------------------------
% BEP (Needs connection to BER)
%-----------------------------
\subsection{Average Bit Error Probability}
Defined as the probability that the reconstructed \gls{symbol} at the receiver output differs from the transmitted binary \gls{symbol}. It is also called \textit{\gls{BER}}. This measure is the most revealing about the nature of the system's reliability and is most often illustrated in system performance evaluations. Considering a binary system, the presence of a bit error causes \gls{symbol} $1$ to be mistaken for \gls{symbol} $0$, or
vice versa.\cite{hayk} 

The conditional (being on a fading channel) Bit Error Probability (BEP) is a non-linear function of instantaneous \gls{SNR}. The nature of non-linearity being a function of the modulation and detection scheme being used.\cite{dcommoha}

Supposing that the conditional BEP is of the form:
$$P_b\left( E|\gamma\right) = C_1 \exp \left( -a_1\gamma\right)$$
as would be for differentially coherent detection of \gls{PSK} or non-coherent detection of orthogonal \gls{FSK}. The average BEP can be written as:
$$P_b\left( E|\gamma\right) \triangleq \int_0^{\infty} P_b\left( E|\gamma\right) p_{\gamma}d\gamma$$
$$=\int_0^{\infty} C_1 \exp \left( -a_1\gamma\right) p_{\gamma}d\gamma$$
The most probable number of bit errors is one given a \gls{symbol} error. Since there are $log_2 M$ bits per
\gls{symbol}, the average probability of \gls{symbol} error with $k$ as an integer is related to the BER as :
$$BER=\frac{2^{k-1}}{2^{k}-1}P_b$$
Bit error ratio of a communication system can be calculated for each \gls{SNR} values as:
$$ \gls{BER}=\frac{\text{number  of  error  bits}}{\text{number  of  transmit  bits}}$$

\subsection{Symbol Rate}
Also known as \textit{baud rate}, is the number of symbol changes or signaling events across the transmission medium per unit time using a digitally modulated signal.

Symbols of various sizes that are quantized and referenced to an \textit{M-ary} alphabet set with which each $n$ \gls{symbol} represents one of the possible discrete amplitude levels. For an \textit{M-ary} system with a \gls{symbol} alphabet containing $M$ statistically independent \gls{symbol}s and a \gls{symbol} duration of $T$ seconds, $\frac{1}{T}$ is its \gls{symbol} rate, expressed in \textit{bauds}\cite{hayk}.

For a communication system, instead of using a binary alphabet with $1$ bit of information per channel \gls{symbol} period, $M$ \gls{symbol}s are used, permitting transmission of $k=log_{2}M$ bits during each \gls{symbol} period. \textit{M-ary} \gls{symbol}s allow a $k$-fold increase in the data rate within the same bandwidth hence their use reduces the required bandwidth by a factor $k$\cite{AWGN}.

\subsection{Channel Utilization}
This is defined as the rate of successful message delivery over a communication channel, also called \textit{throughput}.
The meaningful metric of system performance out of channel utilization is \textit{maximum throughput} which is synonymous with digital bandwidth capacity.

This capacity according to \cite{hayk} is the maximum mutual information $I\left( \mathscr{X;Y} \right)$in any single use of the channel where maximization is over all possible input probability distributions $\left\{ p\left( x_j\right) \right\}$ on $\mathscr{X}$. It is written as: 
$$C = \underset{\left\{ p\left( x_j\right) \right\}}{\max} I\left( \mathscr{X;Y} \right)$$
Measured in \textit{bits per transmission}.

\subsection{Spectral Efficiency}
Also called \textit{bandwidth efficiency}, defined as the ratio of data rate in bits per second to the effectively utilized bandwidth in \cite{hayk}.
$$\rho = \frac{R_b}{B}\ bits/sec/Hz$$
Where $\rho$ is spectral efficiency, $R_b$ is bit rate and $B$ is bandwidth.

Spectral effective modulation techniques that maximize bandwidth efficiency and thus reduce spectral
congestion are used.\cite{AWGN}

\subsection{Latency}
This is defined by \cite{latintel} as the number of clock cycles that are required for the execution core to complete the execution of all micro-ops that form an instruction.

It is the time interval between source stimulation and destination \gls{symbol} change primarily as a consequence of the limited velocity with which any transmission can propagate and delays due to data processing stages along the channel.
The lower limit of latency is determined by the medium being used for communication. Latency caps the maximum rate of information transfer due to the inherent limit on the amount of information that can be in transit at any instance. 

Symbol synchronization between transmitter and receiver can be accomplished in one of several ways. In some
communication systems the transmitter and receiver clocks are synchronized to a master clock, which
provides a very precise timing signal. In this case, the receiver must estimate and compensate for the
relative time delay between the transmitted and received signals. Another method for achieving \gls{symbol}
synchronization is for the transmitter to simultaneously transmit the clock frequency $\frac{1}{T}$ or a multiple of $\frac{1}{T}$ along with the information signal.\cite{Salehi}

%\subsection{Timing Jitter}
%The variation in periodicity of a signal from its target or true frequency otherwise known as jitter is a
%random process. It is the deviation of the time instant at which a given event occurs, relative to a reference time frame, which can be chosen arbitrarily\cite{jitter}.  If there is a slight jitter in
% the position of a sample, the sampling is no longer uniform.
%
% Jitter occurs because carriers are generated using physical devices that have some randomness associated with them. An undesirable characteristic  is caused by electromagnetic interference and cross-talk with carriers of other signals.
% 
%Effects of the jitter is equivalent to frequency modulation (FM) of the baseband signal. A low-level wideband spectral contribution is induced whose properties are very close to those of the quantizing noise. If the  jitter exhibits periodic components the periodic FM will induce low-level spectral lines in the data. Timing jitter can be controlled with very good power supply isolation and stable clock references.\cite{AWGN}
 
