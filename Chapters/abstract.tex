\chapter*{Abstract}

% A signal transmitted over a wireless channel that implements \gls{OFDM} is subjected to several effects that affect its performance. When the channel has multiple reflective paths, signal degradation is more pronounced. Some of the effects which contribute to propagation path loss include interference, attenuation and shadowing.
Design of an \gls{OFDM} system necessitates access to powerful computing hardware, costly proprietary software and programming proficiency on the designer's part. This is because the performance metrics of an \gls{OFDM} system are unavailable for analytical evaluation. These metrics include: \gls{BER} which is a function of \gls{SNR} and \gls{PAPR}. \gls{BER} can only be numerically determined by rigorous calculations involving stochastic processes within a simulation. Without the aforementioned resources, the design process is greatly impeded in the best case, completely hindered in the worst. Whereas a powerful computer can make short work of a complex simulation, the same computer can solve an analytic model even faster. Regardless, even with unencumbered access to computational power, simulation is significantly time-consuming for elaborate models such as an accurate simulation of an \gls{OFDM} system.
 
% The performance of a communication system is evaluated by considering method of transmission, channel parameters and device construction. \gls{BER}, a function of \gls{SNR}, is a central measure of system performance. Additional metrics of performance are \gls{PAPR}, symbol rate, spectral efficiency and latency. Presently, \gls{BER} of an \gls{OFDM} system is determined by measurement after the system has been commissioned or by simulation during the design process. This necessitates access to powerful computing hardware, costly proprietary software and technical expertise on the part of the communication system designer.
%This significantly slows the design process for designers without the aforementioned resources. Even with unencumbered access, simulation is significantly time-consuming for elaborate models such as an accurate model of an \gls{OFDM} system.

This project aims to come up with an analytic model for \gls{BER} performance of \gls{OFDM}  within a fading channel. A mathematical model is a powerful tool in the hands of a system designer, and even more powerful to a designer with access to a computer.

%A powerful computer can make short work of a complex simulation. The same computer can solve an analytic model even faster. This project aims to come up with an analytic model for \gls{BER} performance of \gls{OFDM}  within a fading channel. A mathematical model is a powerful tool in the hands of a system designer, and even more powerful to a designer with access to a computer.

The project will be implemented by creation of a MATLAB simulation model of an \gls{OFDM} system operating over fading channels that have both Rayleigh and Rician distributions. System performance can be obtained from simulation models in form of \gls{BER} curves. Using regression and continuous variation of the simulation model parameters, an expression for a curve of best fit will be obtained.

%This will be achieved by first creating a MATLAB\textcopyright\ simulation model of an \gls{OFDM} system operating over fading channels of both Rayleigh and Rician distribution. The system's \gls{BER} curves under both fading distributions will be plotted and using regression, the expression of a curve of best fit will be established. 
