\chapter*{Abstract}
A signal transmitted over a wireless channel that implements \gls{OFDM} is subjected to several effects that affect its performance. When the channel has multiple reflective paths, signal degradation is more pronounced. Some of the effects which contribute to propagation path loss include interference, attenuation and shadowing.

The performance of a communication system is evaluated by considering method of transmission, channel parameters and device construction. \gls{BER}, a function of \gls{SNR}, is a central measure of system performance. Additional metrics of performance are \gls{PAPR}, symbol rate, spectral efficiency and latency. Presently, \gls{BER} of an \gls{OFDM} system is determined by measurement after the system has been commissioned or by simulation during the design process. This necessitates access to powerful computing hardware, costly proprietary software and technical expertise on the part of the communication system designer. This significantly slows the design process for designers without the aforementioned resources. Even with unencumbered access, simulation is significantly time-consuming for elaborate models such as an accurate model of an \gls{OFDM} system.


A powerful computer can make short work of a complex simulation. The same computer can solve an analytic model even faster. This project aims to come up with an analytic model for \gls{BER} performance of \gls{OFDM}  within a fading channel. A mathematical model is a powerful tool in the hands of a system designer, and even more powerful to a designer with access to a computer.

This will be achieved by first creating a MATLAB\textcopyright\ simulation model of an \gls{OFDM} system operating over fading channels of both Rayleigh and Rician distribution. The system's \gls{BER} curves under both fading distributions will be plotted and using regression, the expression of a curve of best fit will be established. 
