%%%%%%%%%%%%%%%%%%%%%%%%%%%%%%%%%%%%%%%%%%%%%%%
%% - - P R O B L E M   S T A T E M E N T - - %%
%%%%%%%%%%%%%%%%%%%%%%%%%%%%%%%%%%%%%%%%%%%%%%%

\section{Problem Statement}
\gls{OFDM} became the most popular transmission technique for broadband communication due to its advantages such as robustness to multipath fading channels, because of cyclic extension, and high spectral efficiency\cite{wireless_design}. \gls{OFDM} system design revolves around the aforementioned factors in the previous section.

The performance of an \gls{OFDM} system is presently quite difficult to determine analytically due to the probabilistic nature of \gls{BER}. \gls{BER} is a central metric of performance for digital communication systems. This leaves complete implementation and simulation as the only ways through which performance can be measured. As a result, \gls{OFDM} system modeling and performance measuring is the domain of designers and analysts with access to powerful computing hardware for software simulation or costly lab equipment for the same purpose.

Even with access to equipment, simulation is still a computationally intensive activity. Modern computers' processors are powerful but limited. Simulation programs are deeply optimized to run as fast as possible but are still relatively slow compared to the real life implementation. While this delay seems insignificant for a single simulation, the analysis process involves many iterations of the simulation with minor tweaking of system parameters so as to determine the most acceptable compromise.
% Communication in the current world is widely used in different ways such that it is very easy to overlook a great number of its aspects \cite{hayk}. Research on existing communication technologies and invention of new technologies has brought about many variations in existing standards and rise of new standards to the point that it is hard to keep track of various facets of communication systems. For any communication system to be considered for implementation, it must be able have a cutting edge over other existing systems hence need of having a performance analysis on different communication systems.
 
% Performance analysis is a key process that needs to be evaluated. Different factors affecting the system are also supposed to be considered during the analysis with a need to have a practical idea of the system's performance. The process of analyzing should therefore be simple, convenient and economical in application. In relation to \gls{OFDM}, \gls{BER} is the central measure of its systems' performance. Various  \gls{OFDM} standards have already established methods of determining their \gls{BER} performance while still applying different considerations in terms of the channel used and data transmission technique. 

% An \gls{OFDM} variant's performance analysis requires powerful computational abilities in order to perform complex simulations that can provide a \gls{BER} performance model for a fading channel. The project proposed therein is supposed to develop an analytical model in the form of an expression for \gls{BER} performance for an \gls{OFDM} variant that has a Rayliegh channel and Rician fading channel.  