%%%%%%%%%%%%%%%%%%%%%%%%%%%%%%%%%%%%%%%%%%%%%%%%%%%%%%%%%%%%
%% - - P R O B L E M  J U S T I F I C A T I O N - - %%
%%%%%%%%%%%%%%%%%%%%%%%%%%%%%%%%%%%%%%%%%%%%%%%%%%%%%%%%%%%%

\section{Problem Justification}
The process of digital communication system analysis using \gls{OFDM} has been shown to be repetitive and time consuming, even with the necessary hardware. This project aims to come up with an analytic model for \gls{OFDM} system performance, mainly \gls{BER} performance over both Rayleigh and Rician fading channels.

The use of the expression so attained should allow the \gls{OFDM} system analyst to establish a ballpark range for the parameters they expect their system to have, and know the shape of the resulting \gls{BER} curve for various \gls{SNR} levels. Being an analytic technique, this would permit analysis on literal paper with relatively inexpensive computational tools such as scientific calculators.

Moreover, where possible the expression can be evaluated by a computer, at negligible cost in terms of processor time, allowing the designer to power through many iterations with extraordinary time savings as compared to the conventional iterative simulations route.